\documentclass[t,compress,aspectratio=43,12p]{beamer} %字号 frame长宽比
\usepackage[english]{babel}
\usepackage[utf8]{inputenc}

%\usepackage{CJK} % 临时中文支持
\usepackage{CJKutf8} % 临时中文支持 部分\cite报错解决方法
% 请使用\begin{CJK*}{UTF8}{gbsn}你好多戈!\end{CJK*}
% 在gbsn中指定gkai-楷书, gbsn-宋体

\usepackage[uoybglogo]{uoy-beamer}
\usepackage{listings}

%\setbeamerfont{footnote}{size=\tiny} %调整注脚的文字大小
% use \footfullcite{Beacom_2004} to footnote the reference


%%% bibtex package
\usepackage[backend=biber,style=phys,sorting=none]{biblatex}
\addbibresource{reference.bib} %BibTeX数据文件及位置
\AtNextBibliography{\footnotesize} % Bibliography font size调节


% Set to 1 or comment to disable transparency and enable full opacity.
\setblockbodyopacity{0.8}

% Listings
\definecolor{cident}{rgb}{1,0.33,0.42}
\definecolor{ckeyw}{rgb}{1,0.2,0.8}
\definecolor{ccomm}{rgb}{0,0.8,0}
\definecolor{cstr}{rgb}{0.8,0,0}

\lstset{language=[LaTeX]{TeX},
  basicstyle=\footnotesize\ttfamily,
  keywordstyle=\color{ckeyw}\bfseries,
  identifierstyle=\color{cident}\bfseries,
  commentstyle=\color{ccomm},
  stringstyle=\color{cstr},
  showstringspaces=false,
  breaklines=true,
  breakatwhitespace=true,
  tabsize=2,
%   numbers=left,
%   stepnumber=1,
%   firstnumber=1,
%   numberfirstline=true,
  }
  
\uoysetidentity{Department of}{Physics}

\title[Short title]{\begin{CJK*}{UTF8}{gbsn}你好多戈!\end{CJK*}Hello Doge!\\ long fat doge new line}
\subtitle{Subdoge}
\author[Short author(s)]{Meng Meng\inst{1} \and Author Two\inst{2}}
\institute{\inst{1}Sun Yat-sen University \\ \inst{2}Another Affiliation}
\date[18 May]{18 May 2023}

% \setlength{\frametitlemargin}{1mm}

\begin{document}
% ==============================================================
%                         --- Welcome frame
% ==============================================================
\begin{frame}[plain]
\maketitle
\end{frame}

% Uncomment to hide your department next
% \uoysetidentity{}{}

% ==============================================================
%                         --- TOC
% ==============================================================
\begin{frame}
\tableofcontents
\end{frame}



% ==============================================================
%                         --- Section 1
% ==============================================================
\section{Section1} %section show in contents
\begin{frame}[fragile]
\frametitle{Slide title} %slide title show in the top of the slide
\[\Gamma=\frac{1}{\Lambda}\]
\begin{CJK*}{UTF8}{gbsn}你好多戈!\end{CJK*}
%% contents of this slide
This template includes a light and a dark theme, as well as an option to remove the UoY logo from the background. Options:
\begin{description}
  \item[uoybglogo] To place a watermark version of the University of York logo in the background
  \item[darktheme] To enable the dark theme
\end{description}
So for having a dark theme and the logo in the background, you'd load the style with
\begin{lstlisting}
\usepackage[uoybglogo,darktheme]{uoy-beamer}
\end{lstlisting}
\end{frame}

\begin{frame}[fragile]
\frametitle{Slide title}
You may need to add some space after the frame title
\begin{lstlisting}
\setlength{\frametitlemargin}{10mm}
\end{lstlisting}

\pause  %单独成一页,类似动画效果
\vskip 3mm
Using an empty frametitle or not using it, you will need to add initial space by yourself.
To have an ``empty'' frametitle you can use
\begin{lstlisting}
\emptyframetitle
\end{lstlisting}

\pause
or (more portable)
\begin{lstlisting}
\frametitle{\strut}
\end{lstlisting}
\end{frame}

\subsubsection{Section1-2-1}
\begin{frame}[fragile]
\noframetitle
If you do not want any titles in a frame, you can use
\begin{lstlisting}
\noframetitle
\end{lstlisting}
\end{frame}

\begin{frame}[fragile]
\noframetitle[0mm]
Using 
\begin{lstlisting}
\noframetitle[0mm]
\end{lstlisting}

\vskip 5mm
As above, you may want to use the same command with a length to specify where (vertically) the text should begin.
\end{frame}

\begin{frame}[fragile]
\framesingletitle{Single title}
If you need just a simple title you can use
\begin{lstlisting}
\framesingletitle{your title}
\end{lstlisting}

\vskip 5mm
You may want to use the same command with a length to specify where (vertically) the title should begin.
\begin{lstlisting}
\framesingletitle[0mm]{your title}
\end{lstlisting}
\end{frame}

\subsection{Section1-1}
\begin{frame}[fragile]
\frametitle{Blocks}
\begin{block}{Additional blocks}
\begin{itemize}
  \item termblock
  \item problock
  \item conblock
  \item yellowblock
  \item blueblock
\end{itemize}
\end{block}
Transparency of blocks may not look very nice on dark backgrounds, to disable it:
\begin{lstlisting}
\setblockbodyopacity{1}
\end{lstlisting}
\end{frame}

\subsection{Section1-2}
\subsubsection{Section1-2-1}
\begin{frame}
\frametitle{Additional blocks}
\begin{block}{block}
\begin{itemize} \item text \end{itemize}
\end{block}
\begin{block}[green]{block[green] - all blocks support custom colours for item elements}
\begin{itemize} \item text \end{itemize}
\end{block}
\begin{exampleblock}{exampleblock}
\begin{itemize} \item text \end{itemize}
\end{exampleblock}
\begin{alertblock}{alertblock}
\begin{itemize} \item text \end{itemize}
\end{alertblock}
\end{frame}

\begin{frame}
\frametitle{Additional blocks}
\begin{blueblock}{blueblock}
\begin{itemize} \item text \end{itemize}
\end{blueblock}
\begin{yellowblock}{yellowblock}
\begin{itemize} \item text \end{itemize}
\end{yellowblock}
\begin{termblock}{termblock}
\begin{itemize} \item text \end{itemize}
\end{termblock}
\end{frame}

\begin{frame}
\frametitle{Additional blocks}
\begin{cleanblock}{cleanblock}
\begin{itemize} \item text \end{itemize}
\end{cleanblock}
\begin{whiteblock}{whiteblock}
\begin{itemize} \item text \end{itemize}
\end{whiteblock}
\begin{grayblock}{grayblock}
\begin{itemize} \item text \end{itemize}
\end{grayblock}
\end{frame}

\begin{frame}
\frametitle{Additional blocks}
\begin{problock}{problock}
\begin{itemize} \item text \end{itemize}
\end{problock}
\begin{conblock}{conblock}
\begin{itemize} \item text \end{itemize}
\end{conblock}
\end{frame}

\begin{frame}
\frametitle{Size always matters}
\begin{block}{block - long}
  \begin{itemize}
    \item blabla...
    \item blabla...
    \begin{itemize}
      \item nested blabla...
      \item nested blabla...
    \end{itemize}
    \item blabla...
    \begin{itemize}
    \item nested blabla...
    \item nested blabla...
    \item nested blabla...
    \end{itemize}
    \item blabla...
    \item blabla...
  \end{itemize}
\end{block}
\begin{block}{block - empty}
\end{block}
\end{frame}

\begin{frame}
\frametitle{Size always matters}
\begin{block}{block - longest}
  \begin{itemize}
    \item blabla...
    \item blabla...
    \begin{itemize}
      \item nested blabla...
      \item nested blabla...
    \end{itemize}
    \item blabla...
    \begin{itemize}
    \item nested blabla...
    \item nested blabla...
    \item nested blabla...
    \end{itemize}
    \item blabla...
    \item blabla...
    \item blabla...
    \begin{itemize}
    \item nested blabla...
    \item nested blabla...
    \item nested blabla...
    \item nested blabla...
    \end{itemize}
  \end{itemize}
\end{block}
\end{frame}

\subsection{Section1-3}
\begin{frame}
\frametitle{This is an example of a very long title that needs to be split in two lines (no longer than 2 lines are visually comfortable)}
Bla bla bla...
\end{frame}



% ==============================================================
%                         --- Section 2
% ==============================================================
\section{Section2}
\begin{frame}
\frametitle{Slide title}
Bla bla bla...
\end{frame}

\subsection{Section2-1}
\begin{frame}
\frametitle{Slide title}
Bla bla bla...
\end{frame}

\subsection{Section2-2}
\begin{frame}
\frametitle{Slide title}
Bla bla bla...
\end{frame}


% ==============================================================
%                         --- THE END
% ==============================================================
\begin{frame}
\framesingletitle{The END}

\begin{center}
\begin{minipage}{0.8\textwidth}
  \quotesentence{%
  ``Do not close a presentation with The End'' [author]
  }
\end{minipage}
\end{center}
\end{frame}

\begin{frame}
  \printbibliography
\end{frame}





% -------------------------------------------------------------
\end{document}
